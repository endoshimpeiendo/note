%%%%%%%%%%%%%%%%%%%%%%%%%%%%%%%%%%%%%%%%%
% Article Notes
% LaTeX Template
% Version 1.0 (1/10/15)
%
% This template has been downloaded from:
% http://www.LaTeXTemplates.com
%
% Authors:
% Vel (vel@latextemplates.com)
% Christopher Eliot (christopher.eliot@hofstra.edu)
% Anthony Dardis (anthony.dardis@hofstra.edu)
%
% License:
% CC BY-NC-SA 3.0 (http://creativecommons.org/licenses/by-nc-sa/3.0/)
%
%%%%%%%%%%%%%%%%%%%%%%%%%%%%%%%%%%%%%%%%%

%----------------------------------------------------------------------------------------
%	PACKAGES AND OTHER DOCUMENT CONFIGURATIONS
%----------------------------------------------------------------------------------------

\documentclass[
10pt, % Default font size is 10pt, can alternatively be 11pt or 12pt
a4paper, % Alternatively letterpaper for US letter
twocolumn, % Alternatively onecolumn
landscape % Alternatively portrait
]{article}

\input{structure.tex} % Input the file specifying the document layout and structure
\usepackage{hyperref}


%----------------------------------------------------------------------------------------
%	ARTICLE INFORMATION
%----------------------------------------------------------------------------------------

\articletitle{Moderate Modal Realism} % The title of the article
\articlecitation{\cite{Miller2001ModerateModalRealism}} % The BibTeX citation key from your bibliography

\datenotesstarted{December 16, 2018} % The date when these notes were first made
\docdate{\datenotesstarted; rev. \today} % The date when the notes were lasted updated (automatically the current date)
\docauthor{Summarized and Commented by Shimpei Endo} % Your name

%----------------------------------------------------------------------------------------

\begin{document}

\pagestyle{myheadings} % Use custom headers
\markright{\doctitle} % Place the article information into the header

%----------------------------------------------------------------------------------------
%	PRINT ARTICLE INFORMATION
%----------------------------------------------------------------------------------------

\thispagestyle{plain} % Plain formatting on the first page

\printtitle % Print the title

%----------------------------------------------------------------------------------------
%	ARTICLE NOTES
%----------------------------------------------------------------------------------------

\section*{In a nutshell... }
From a modal realist standpoint, \cite{Miller2001ModerateModalRealism} suggests to downsize or \emph{moderate} modal realism.
First, Miller refutes two rival theories (sec.2 for linguistic ersatzism and sec. 3 for fictionalism).
After clarifying his standard for ontological economy (sec. 4), Miller suggests a downsized and moredate, or \emph{hybrid} version of modal realism with respect to ontologies about \emph{outer sphere}, which the original Lewis's variant push us unnecessary and harmful constraints.  

\noindent \textbf{Keywords:} modal realism

\section*{Comments by me}
This work shows the most coolest direction among countless possible worlds industry.


\paragraph{Towards thinner modal realisms.}
Miller, a claimed modal realist, sees his rivals (viz., linguistic ersatzism and fictionalism) are more problematic than modal realism.
Miller argues that ``the main flaw of Modal Realism, its ontological extravagance, can be completeley eliminated without any loss of explanatory power.'' (p. 3)
In other words, this paper aims at the possibility of non-Lewisian modal realism. In his own words:

\begin{quote}
  How would Modal Realism look if developed by philosophers who disagreed with Lewis about nominalism, personal identity, minds, reductionism, or materialism? Such interesting and potentially fruitful questions go unasked when we persist in thinking of Modal Realism as the idiosyncrasy of one brilliant mind. (p. 3-4)
\end{quote}

\section{Epistemological worries about modal realism}

\paragraph{Epistemological worries.}
 Miller begins with undesired aspects of modal realism, focusing on \emph{epitemological worries}, which question how we do know things going on other possible worlds.
 A classical defence says that we know many things to which we have no empirical access (e.g. sets, numbers).

 Miller is unsatisfied with this response because it forces us to exclude empiricist epistemology from the entire modal realism camp.
 Miller in fact defends a view that our knowledge of possibilia is empirical (p. 5-6), citing examples of quantum mechanics.

 \paragraph{Island Universe problem.}
 The next concern Miller addresses is the Island Universe problem (Bigelow 1987) \cite{Bigelow1987}.


\section{What's wrong with ersatzism?}
The following two sections explain why Miller wants to defend modal realism by showing how (further) problematic its rivals are. This section (p. 8-15) deals with (linguistic) ersatzism.

\section{What's wrong with fictionalism?}
Succeeding from the previous section, this section (p. 15 - 20) disputes how bad fictionalism ends up.

\section{Modal realism and ontological extravagance}
Before ``downsizing'' modal realism, Miller settles his standard for evaluating ontological economy: ``the vice of extravafance and the virtue of parsimony'' (p. 21).
Miller basically follows \cite{Lewis1986}'s criterion: the number of types i.e. \emph{qualitative parsimony} matters instead of tokens.

\paragraph{Modal realism is not maximally unparsimonious.}
Joseph Melia, for instance, attacks Lewis for him being the worst for ontological economy.

\begin{quote}
``For instance, Lewis commited to the unicorns, to the gods, to the ghosts, to the qualia which occur in other possible worlds. Indeed, Lewis is commited t oevery possible kind of thing. Lewis's theory is as \textit{qualitatively} unparsimonious as any consistent theory could be.'' (Melia 1992, p. 192)
\end{quote}

Miller calls attention to two points to defend modal realism from Melia's horn.
First, Lewis' possible worlds contain a fewer kinds of things: anything is concrete, existing in time and space.
So comparing to other ontologies featuring abstract entities, Lewis is more austere.
Second, seemingly abundant entities (e.g. unicorns) do not require new types if modal realists can reduce it into an existing (and admitted) type. For instance, a scientific realist would reduce unicorns into collections of subatomic particules.

The key is that Miller here (p. 25) does not intend to rescue \emph{any} kind of modal realism. Rather, Miller secures the survival of \emph{some} variants of modal realism.
Miller's attitude within modal realims is relaxed and welcoming (p. 25).

\section{Ontology and the outer sphere}
This section (p.25-30) introduces the discussion about the outer sphere, which Miller claims is an unnecessary constraint which Lewis asks for modal realist.

\section{Realism without an outer sphere}
This section depicts a version of modal realism which exclude an outer sphere.

\section{Problems for the hybrid theory?}
Moderate modal realism, Miller's concluding position, is a ``hybrid theory'' in a sense that it is realist about the inner sphere but fictionalist about the outer sphere (p. 32). The two tasks in this section are to show (i) moderate modal realims is free from problems of the original fictionalism and (ii) fictionalism about outer sphere successfuly offer truth conditions for modal sentences.

\section{Conclusing remarks.}
The last section wraps up the argument so far.
Miller suggests to moderate (i.e. downsize) modal realism by cancelling some unnecessary and problematic constraints which David Lewis asked us modal realists.
This move has merit of controlling extravagance without a loss of explanatory power.

\begin{quote}
  Moderate Modal Realism is certainly less colorful than its more outrageous cousin, but it is not far that reason less worthy of consideration. (p. 36, the very closing sentence of this paper)
\end{quote}



%----------------------------------------------------------------------------------------
%	BIBLIOGRAPHY
%----------------------------------------------------------------------------------------

%\renewcommand{\refname}{Reference} % Change the default bibliography title

\bibliography{Mendeley} % Input your bibliography file

%----------------------------------------------------------------------------------------

\end{document}
