%%%%%%%%%%%%%%%%%%%%%%%%%%%%%%%%%%%%%%%%%
% Article Notes
% LaTeX Template
% Version 1.0 (1/10/15)
%
% This template has been downloaded from:
% http://www.LaTeXTemplates.com
%
% Authors:
% Vel (vel@latextemplates.com)
% Christopher Eliot (christopher.eliot@hofstra.edu)
% Anthony Dardis (anthony.dardis@hofstra.edu)
%
% License:
% CC BY-NC-SA 3.0 (http://creativecommons.org/licenses/by-nc-sa/3.0/)
%
%%%%%%%%%%%%%%%%%%%%%%%%%%%%%%%%%%%%%%%%%

%----------------------------------------------------------------------------------------
%	PACKAGES AND OTHER DOCUMENT CONFIGURATIONS
%----------------------------------------------------------------------------------------

\documentclass[
10pt, % Default font size is 10pt, can alternatively be 11pt or 12pt
a4paper, % Alternatively letterpaper for US letter
twocolumn, % Alternatively onecolumn
landscape % Alternatively portrait
]{article}

%%%%%%%%%%%%%%%%%%%%%%%%%%%%%%%%%%%%%%%%%
% Article Notes
% Structure Specification File
% Version 1.0 (1/10/15)
%
% This file has been downloaded from:
% http://www.LaTeXTemplates.com
%
% Authors:
% Vel (vel@latextemplates.com)
% Christopher Eliot (christopher.eliot@hofstra.edu)
% Anthony Dardis (anthony.dardis@hofstra.edu)
%
% License:
% CC BY-NC-SA 3.0 (http://creativecommons.org/licenses/by-nc-sa/3.0/)
%
%%%%%%%%%%%%%%%%%%%%%%%%%%%%%%%%%%%%%%%%%

%----------------------------------------------------------------------------------------
%	REQUIRED PACKAGES
%----------------------------------------------------------------------------------------

\usepackage[includeheadfoot,columnsep=2cm, left=1in, right=1in, top=.5in, bottom=.5in]{geometry} % Margins

\usepackage[T1]{fontenc} % For international characters
\usepackage{XCharter} % XCharter as the main font

\usepackage{natbib} % Use natbib to manage the reference
\bibliographystyle{apalike} % Citation style

\usepackage[english]{babel} % Use english by default

%----------------------------------------------------------------------------------------
%	CUSTOM COMMANDS
%----------------------------------------------------------------------------------------

\newcommand{\articletitle}[1]{\renewcommand{\articletitle}{#1}} % Define a command for storing the article title
\newcommand{\articlecitation}[1]{\renewcommand{\articlecitation}{#1}} % Define a command for storing the article citation
\newcommand{\doctitle}{\articlecitation\ --- ``\articletitle''} % Define a command to store the article information as it will appear in the title and header

\newcommand{\datenotesstarted}[1]{\renewcommand{\datenotesstarted}{#1}} % Define a command to store the date when notes were first made
\newcommand{\docdate}[1]{\renewcommand{\docdate}{#1}} % Define a command to store the date line in the title

\newcommand{\docauthor}[1]{\renewcommand{\docauthor}{#1}} % Define a command for storing the article notes author

% Define a command for the structure of the document title
\newcommand{\printtitle}{
\begin{center}
\textbf{\Large{\doctitle}}

\docdate

\docauthor
\end{center}
}

%----------------------------------------------------------------------------------------
%	STRUCTURE MODIFICATIONS
%----------------------------------------------------------------------------------------

\setlength{\parskip}{3pt} % Slightly increase spacing between paragraphs

% Uncomment to center section titles
%\usepackage{sectsty}
%\sectionfont{\centering}

% Uncomment for Roman numerals for section numbers
%\renewcommand\thesection{\Roman{section}}
 % Input the file specifying the document layout and structure
\usepackage{hyperref}


%----------------------------------------------------------------------------------------
%	ARTICLE INFORMATION
%----------------------------------------------------------------------------------------

\articletitle{The Cardinality Objection to David Lewis's Modal Realism} % The title of the article
\articlecitation{\cite{Pruss2001_CardinalityObjection}} % The BibTeX citation key from your bibliography

\datenotesstarted{December 25, 2018} % The date when these notes were first made
\docdate{\datenotesstarted; rev. \today} % The date when the notes were lasted updated (automatically the current date)
\docauthor{Summarized and Commnted by Shimpei Endo} % Your name

%----------------------------------------------------------------------------------------

\begin{document}

\pagestyle{myheadings} % Use custom headers
\markright{\doctitle} % Place the article information into the header

%----------------------------------------------------------------------------------------
%	PRINT ARTICLE INFORMATION
%----------------------------------------------------------------------------------------

\thispagestyle{plain} % Plain formatting on the first page

\printtitle % Print the title

%----------------------------------------------------------------------------------------
%	ARTICLE NOTES
%----------------------------------------------------------------------------------------

\section*{In a nutshell... }
Pruss presents the \emph{cardinality} objection to Lewisian modal realsim.
Pruss updates previous cardinality objections of Forrest and Armstrong. Pruss demonstrates that unrestricted principle of recombination entails that the collection of all possible worlds cannot be a set.
Pruss also notes that any theory of possibilia cannot say that the collection of all possible worlds is a set.

\noindent \textbf{Keywords:} modal realism, anti-Lewis, David Lewis, cardinality

\section*{Comments by me}
\paragraph{System objection.}
According to my own categorization, the cardinality objection is categorized as ``system'' objection for it points out an inner contradiction within Lewis's theory.


\section*{1. Introduction}
\paragraph{The original cardinality objection by Forrest and Armstrong.}
\cite{Forrest1984} argues that ``by considering an allegedly possible world which has as its parts duplicates of all the worlds one can arrive at a contradiction to the effect that the number of individuals in the totality of all worlds has cardinality greater than it has'' [p. 169, in Pruss's words].

\paragraph{Lewis's own defense: unrestricted principle of recombination.}

\begin{quote}
  Not only two possible individuals, but any number should admit of combination by means of coexisting duplicates. Indeed, the number might be infinite.
  \cite[p. 89]{Lewis1986}
\end{quote}

Responding to Forrest and Armstrong, Lewis further added another restriction ``size and shape permitting'' [\textit{ibid}] to block ``Forrest and Armstrong's gigantic world which is an aggregation of copies of all worlds.'' [p. 170]

\paragraph{Nolan's defense from cardinality objection.}
Nolan defended Lewisian modal realism by imposing the new principle of \emph{unrestricted} principle of recombination [p. 170].

\section*{2. There is no set of all possible worlds}
This section offers Pruss's version of cardinality objection.
The claim is there is no set of all possible worlds.
For RAA, assume otherwise.
Take $n$ such that  $n > | w |$.
By Axiom of Choice, there is $\aleph_0 = n^* \meq n > | w |$.
This indicates that ther exists exactly $n$ distinct cardinal numbers $m$ satisfying $\alph_0 leq m < n^*$.
Recall that unrestricted recombination tells that for any $m$, there is a possible world $w_m$ which contains $m$ photons.
As far as $m \neq m^*$, we can have $w_m \neq w_{m^{*}}$ since they contain different numbers of photons. So
there is at least $n$ worlds which satisfies
$\aleph_0 \leq m \leq n^* $.
This contradict the beginning assumption of $n > | w |$.

\paragraph{Pro-choice: Pruss's cost-benefit calicuration}
The argument above relies on the Axiom of Choice (AC). Noticing that AC leads some paradoxical conclusions such as Banach-Tarski's one, Pruss weights AC to EMR.
``If we were forced to give up the AC in order to hold on to Lewis's EMR, the price would be too high.'' [p. 172]

For blocking EMR, the argument above can be reconstructed after replacing AC to a reasonable assumption as follow:

\begin{quote}
  There is a well-defined predicate $S$ such that $Sw$ holds if and only if $w$ is a possible world such that the collection of all photons in w is a set, so that if $A$ is a set, so is $\{ x \in A : Sx \}$
\end{quote}

\section*{3. Lewis's proviso}
This section deals with Lewis's restriction which only allows recombination as ``size and shape'' permitting. Pruss claims that the contradiction still remains after this limitation.

\section*{4. What should Lewis do?}
\paragraph{Abandon EMR.}
Pruss advises Lewis just to abandon EMR because his concrete view on possible worlds leads us to think there is a set of possible worlds (other concrete things can form sets, with no problem). If not, Lewis at least confirm that the collection of possible worlds is a \emph{proper class}, instead of a set. [p. 175]

\paragraph{What about Nolan?}
Pruss further discuss a weaker variant of Nolan.

\paragraph{The bottom line: there is no set of all possible worlds.}
In closing, Pruss expands his argument to any reasonable theory of possibilia; they should not have a set of all possible worlds.
[p. 176]
%----------------------------------------------------------------------------------------
%	BIBLIOGRAPHY
%----------------------------------------------------------------------------------------

%\renewcommand{\refname}{Reference} % Change the default bibliography title

\bibliography{Mendeley} % Input your bibliography file

%----------------------------------------------------------------------------------------

\end{document}
