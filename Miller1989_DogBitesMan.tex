%%%%%%%%%%%%%%%%%%%%%%%%%%%%%%%%%%%%%%%%%
% Article Notes
% LaTeX Template
% Version 1.0 (1/10/15)
%
% This template has been downloaded from:
% http://www.LaTeXTemplates.com
%
% Authors:
% Vel (vel@latextemplates.com)
% Christopher Eliot (christopher.eliot@hofstra.edu)
% Anthony Dardis (anthony.dardis@hofstra.edu)
%
% License:
% CC BY-NC-SA 3.0 (http://creativecommons.org/licenses/by-nc-sa/3.0/)
%
%%%%%%%%%%%%%%%%%%%%%%%%%%%%%%%%%%%%%%%%%

%----------------------------------------------------------------------------------------
%	PACKAGES AND OTHER DOCUMENT CONFIGURATIONS
%----------------------------------------------------------------------------------------

\documentclass[
10pt, % Default font size is 10pt, can alternatively be 11pt or 12pt
a4paper, % Alternatively letterpaper for US letter
twocolumn, % Alternatively onecolumn
landscape % Alternatively portrait
]{article}

\input{structure.tex} % Input the file specifying the document layout and structure
\usepackage{hyperref}


%----------------------------------------------------------------------------------------
%	ARTICLE INFORMATION
%----------------------------------------------------------------------------------------

\articletitle{Dog Bites Man: A Defense of Modal Realism} % The title of the article
\articlecitation{\cite{Miller1989DogBitesMan}} % The BibTeX citation key from your bibliography

\datenotesstarted{December 24, 2018} % The date when these notes were first made
\docdate{\datenotesstarted; rev. \today} % The date when the notes were lasted updated (automatically the current date)
\docauthor{Summarized and Commnted by Shimpei Endo} % Your name

%----------------------------------------------------------------------------------------

\begin{document}

\pagestyle{myheadings} % Use custom headers
\markright{\doctitle} % Place the article information into the header

%----------------------------------------------------------------------------------------
%	PRINT ARTICLE INFORMATION
%----------------------------------------------------------------------------------------

\thispagestyle{plain} % Plain formatting on the first page

\printtitle % Print the title

%----------------------------------------------------------------------------------------
%	ARTICLE NOTES
%----------------------------------------------------------------------------------------

\section*{In a nutshell... }
Miller, a huge fan of David Lewis's modal realism, refuses Lycan's charge accusing that Lewis also leaves primitive modality as their rivals.

\noindent \textbf{Keywords:} modal realism, parity, David Lewis, Lycan,

\section*{Comments by me}
\paragraph{Pro-Lewis!}
Miller is one of few instance who does not hide his affection and support to David Lewis.
\paragraph{Check further responses.}
Find the paper Lycan initiated this dispute.
Also, see Lycan's response and a further response from Miller.
\paragraph{Words!}
Wording of Miller and Lycan is attractive (e.g. ``mad dog realism'', ``a kettle vs. a pot''.).
Copy them.

\section*{Detailed contents}

\paragraph{Defend Lewis from Lycan!}
This short (3 pages in length) paper defends Lewisian modal realism from William G. Lycan's attack accusing Lewisian --what Lycan calls ``mad dog''-- modal realism still contains primitive even though Lewis appeals the full reduction is the point to accept his version.
``the pot alleges the kettle to be black. I wish to defend the kettle'' [p. 476].

\paragraph{Naylor-Yagisawa's mutatis mutandis argument for impossibilia.}
Miller briefly mentions arguments of Yagisawa and Naylor to emphasize how (more) important Lycan's objection is.
Naylor and Yagisawa both claim that Lewis should accept not only possible worlds and possibilia but also impossible worlds and impossibilia for his same argument.
While Lewis' original response is too brief to calm down these complains, Miller cites Sharlow's argument to rescue Lewis.
According to Sharlow, possible worlds are already enough to capture both possibilia and impossibilia, so there is no need for expanding modal realism to cover impossible worlds.

\paragraph{Lycan's argument.}

\paragraph{Miller evaluates Lycan more deadly to Lewis.}
According to Miller, Naylor-Yagisawa's line is a ``mere embrassment'' [p. ] because it just claims that Lewis is \emph{incomplete} and embraces it in the end. They just ask to add one more.

In contrast, Lycan's argument is crucial, Miller evaluates, to Lewis' entire project.
Because Lewis admits that his possible worlds are extras for ontological economy, possible worlds should be indispensable for Lewis. Otherwise, he should expel such unnecessary components for the very standard Lewis himself adopts.

\paragraph{Lewis' possible worlds may contain some modality, but only unimportant ones and not all.}

\paragraph{Closing remark.}
\begin{quotation}
  Let me add that I defend Lewis's position because I think it is correct. Incredulous stares to the contrary, there are at least two Mad Dog Modal Realists. I would also like to note that Lewis's objection to Ersatz Modal Realism still stands. Lycan's objection about impossible worlds fails against Lewis, and even if it succeeded would not save Ersatz Modal Realism. I have shown that Mad Dog Modal Realists do not have to drag in any modal primitive. \emph{The kettle is shining, what about the pot?} [p. 478, emphasis by Endo]
\end{quotation}

%----------------------------------------------------------------------------------------
%	BIBLIOGRAPHY
%----------------------------------------------------------------------------------------

%\renewcommand{\refname}{Reference} % Change the default bibliography title

\bibliography{Mendeley} % Input your bibliography file

%----------------------------------------------------------------------------------------

\end{document}
