%%%%%%%%%%%%%%%%%%%%%%%%%%%%%%%%%%%%%%%%%
% Article Notes
% LaTeX Template
% Version 1.0 (1/10/15)
%
% This template has been downloaded from:
% http://www.LaTeXTemplates.com
%
% Authors:
% Vel (vel@latextemplates.com)
% Christopher Eliot (christopher.eliot@hofstra.edu)
% Anthony Dardis (anthony.dardis@hofstra.edu)
%
% License:
% CC BY-NC-SA 3.0 (http://creativecommons.org/licenses/by-nc-sa/3.0/)
%
%%%%%%%%%%%%%%%%%%%%%%%%%%%%%%%%%%%%%%%%%

%----------------------------------------------------------------------------------------
%	PACKAGES AND OTHER DOCUMENT CONFIGURATIONS
%----------------------------------------------------------------------------------------

\documentclass[
10pt, % Default font size is 10pt, can alternatively be 11pt or 12pt
a4paper, % Alternatively letterpaper for US letter
twocolumn, % Alternatively onecolumn
landscape % Alternatively portrait
]{article}

%%%%%%%%%%%%%%%%%%%%%%%%%%%%%%%%%%%%%%%%%
% Article Notes
% Structure Specification File
% Version 1.0 (1/10/15)
%
% This file has been downloaded from:
% http://www.LaTeXTemplates.com
%
% Authors:
% Vel (vel@latextemplates.com)
% Christopher Eliot (christopher.eliot@hofstra.edu)
% Anthony Dardis (anthony.dardis@hofstra.edu)
%
% License:
% CC BY-NC-SA 3.0 (http://creativecommons.org/licenses/by-nc-sa/3.0/)
%
%%%%%%%%%%%%%%%%%%%%%%%%%%%%%%%%%%%%%%%%%

%----------------------------------------------------------------------------------------
%	REQUIRED PACKAGES
%----------------------------------------------------------------------------------------

\usepackage[includeheadfoot,columnsep=2cm, left=1in, right=1in, top=.5in, bottom=.5in]{geometry} % Margins

\usepackage[T1]{fontenc} % For international characters
\usepackage{XCharter} % XCharter as the main font

\usepackage{natbib} % Use natbib to manage the reference
\bibliographystyle{apalike} % Citation style

\usepackage[english]{babel} % Use english by default

%----------------------------------------------------------------------------------------
%	CUSTOM COMMANDS
%----------------------------------------------------------------------------------------

\newcommand{\articletitle}[1]{\renewcommand{\articletitle}{#1}} % Define a command for storing the article title
\newcommand{\articlecitation}[1]{\renewcommand{\articlecitation}{#1}} % Define a command for storing the article citation
\newcommand{\doctitle}{\articlecitation\ --- ``\articletitle''} % Define a command to store the article information as it will appear in the title and header

\newcommand{\datenotesstarted}[1]{\renewcommand{\datenotesstarted}{#1}} % Define a command to store the date when notes were first made
\newcommand{\docdate}[1]{\renewcommand{\docdate}{#1}} % Define a command to store the date line in the title

\newcommand{\docauthor}[1]{\renewcommand{\docauthor}{#1}} % Define a command for storing the article notes author

% Define a command for the structure of the document title
\newcommand{\printtitle}{
\begin{center}
\textbf{\Large{\doctitle}}

\docdate

\docauthor
\end{center}
}

%----------------------------------------------------------------------------------------
%	STRUCTURE MODIFICATIONS
%----------------------------------------------------------------------------------------

\setlength{\parskip}{3pt} % Slightly increase spacing between paragraphs

% Uncomment to center section titles
%\usepackage{sectsty}
%\sectionfont{\centering}

% Uncomment for Roman numerals for section numbers
%\renewcommand\thesection{\Roman{section}}
 % Input the file specifying the document layout and structure
\usepackage{hyperref}


%----------------------------------------------------------------------------------------
%	ARTICLE INFORMATION
%----------------------------------------------------------------------------------------

\articletitle{What is Spatial Logic?} % The title of the article
\articlecitation{} % The BibTeX citation key from your bibliography

\datenotesstarted{December 15, 2018} % The date when these notes were first made
\docdate{\datenotesstarted; rev. \today} % The date when the notes were lasted updated (automatically the current date)
\docauthor{Summarized and Commnted by Shimpei Endo} % Your name

%----------------------------------------------------------------------------------------

\begin{document}

\pagestyle{myheadings} % Use custom headers
\markright{\doctitle} % Place the article information into the header

%----------------------------------------------------------------------------------------
%	PRINT ARTICLE INFORMATION
%----------------------------------------------------------------------------------------

\thispagestyle{plain} % Plain formatting on the first page

\printtitle % Print the title

%----------------------------------------------------------------------------------------
%	ARTICLE NOTES
%----------------------------------------------------------------------------------------

\section*{In a nutshell... }
This introductory chapter of \cite{VanBenthem2007} overview \emph{spatial logic}, containing its description and historcial contexts.
It has mattered to spatial logic (as other logics) to take a \emph{balance} between \emph{expressive power} and \emph{computational complexity}.

\noindent \textbf{Keywords:} spatial logic, Hilbert, Tarski,

\section*{Definiton and analogy.}
\paragraph{Definition of spatial logic.}
``By a \textit{spatial logic}, we understand any formal language interpreted over a class of structures featurung geometrical entities and relations, broadly construed.'' (p.1)
Note that the authors do \emph{limit} it to a certain kind of `spaces'. Any `space', say topological, affine, metric, or more specific ones like Euclidean  three dimensional space are all what spatial logics matter.

\paragraph{Pararell between space and time.}
It is helpful to see the parellels between space (spatial logic) and time (temporal logic), the more established area.
Temporal logic has its appeal in its good balance between expressive power and computational complexity. (p. 2)
So expected is spatial logic.

\section*{Historical Contexts: Hilbert and Tarski.}
\paragraph{Historical contexts: Hilbert.}
Euclidean classical geometry was a target of Hilbert's mathematical analysis.
Hilbert relied on (abstract and lightly mathematicalized but still) idiomatic German for its language.

\paragraph{Historical contexts: Tarski.}
Spaces needed to wait Tarski 1959 for its being completely formalized. The development of formal logic and model-theoretic semantics made possible to ``prove the precise inferential and expressive resources of geometry'' (p. 2).
The aim behind his 1959 paper was to search consequences of restricing the syntax (particularly, of a first-order logic).
Sacrificing its expressive power, Tarski found that the theory of elementary geometry is \emph{decidable}, meaning that a mechanical method exists to determine whether a given sentence is true under the intended interpretation. Note that the second-order theory required for all Hilbert's axioms is undecidable.

\paragraph{Tarski's impact.}
``Tarski's discovery illustrates the most distinctive feature of logic in the wake of the model-theoretic revolution of the previous century: its fundamentally linguistic orientation.'' (p. 3)
``On this view, spatial logic, as defined above, becomes the study of the relationship between geometrical structures and the spatial languages which describe them.'' (p.3)

\paragraph{Elementary geometry aximatized }
In elementary geometry, objects can be defined by a fixed number of points in points in the Euclidean plane. But we want more: polygons, arbitrary connected regions and what Tarski's Geometry of Solids dealed. Tarski's own system uses the second-order syntax and the object variables are assigned to not point but certain regions called solids (more specifically, the regular closed subsets of $\mathbb{R}^3$).
The result is remarkable. The resulting theory (i) can be completely axiomatized in the second-order language and (ii) is categorical.
\footnote{A theory is categorical when any model of the teory is isomorphic to the normal interpretation on the reals.}

\paragraph{More coarse: binding topology}
Tarski and McKinsey 1944 found that ``topology has small decidable fragments'' once interpretating topological interior as a modal operator.

\section*{Three princpal directions of spatial logic}
\paragraph{1. What geometrical entities do you use for interpreatation?}

\paragraph{2. What do you assign to the non-logical primitives?}
Like $|| p ||$.
\paragraph{3. Which syntax?}

\paragraph{Classification tradition of geometries, connected.}
Spatial logic clasifies geometrical languages with respect to their spatial primitives. This shares the same idea with the long-standing tradition of classification of geometries (e.g. Klein's Erlanger Programm).
However, the difference is ``the logical approach opens up many new possibilities in this regard, such as, for instance, a new sort of invariance between topological patterns, much coarser even than topological homeomorphism, viz. modal bisimulation.'' (p. 5)

\section*{Four salient issues of spatial logic}
\paragraph{1. How can we characterize its valid formulas?}
\paragraph{2. What is its expressive power?}
\paragraph{3. What is its computational complexity?}
\paragraph{4. Whata laternative interpretations does it have?}

\section*{Recent trends: from computer science.}
\begin{description}
  \item [Qualitative spatial reasoning.]
  \item [Spatial database.]
  \item [Image processing.]
\end{description}

\section*{To a misunderstanding.}
\paragraph{A possible misunderstanding.}
\paragraph{A reply (with an aid of historical backgrounds)}

\section*{Construction of this volume.}

\begin{description}
  \item [A. Tarskian spirit: first-order languages.]
  \item [B. Modal logics for topology: Framgement of first order logic.]
  \item [C. Space combined with others: time.]
  \item [D. Further mathematical and computational advance.]
  \item [E. A coda: metaphysics. ]
\end{description}

\section*{What does \emph{not} this book talk about}

\section*{Comments by me}
%----------------------------------------------------------------------------------------
%	BIBLIOGRAPHY
%----------------------------------------------------------------------------------------

%\renewcommand{\refname}{Reference} % Change the default bibliography title

\bibliography{Mendeley} % Input your bibliography file

%----------------------------------------------------------------------------------------

\end{document}
