%%%%%%%%%%%%%%%%%%%%%%%%%%%%%%%%%%%%%%%%%
% Article Notes
% LaTeX Template
% Version 1.0 (1/10/15)
%
% This template has been downloaded from:
% http://www.LaTeXTemplates.com
%
% Authors:
% Vel (vel@latextemplates.com)
% Christopher Eliot (christopher.eliot@hofstra.edu)
% Anthony Dardis (anthony.dardis@hofstra.edu)
%
% License:
% CC BY-NC-SA 3.0 (http://creativecommons.org/licenses/by-nc-sa/3.0/)
%
%%%%%%%%%%%%%%%%%%%%%%%%%%%%%%%%%%%%%%%%%

%----------------------------------------------------------------------------------------
%	PACKAGES AND OTHER DOCUMENT CONFIGURATIONS
%----------------------------------------------------------------------------------------

\documentclass[
10pt, % Default font size is 10pt, can alternatively be 11pt or 12pt
a4paper, % Alternatively letterpaper for US letter
twocolumn, % Alternatively onecolumn
landscape % Alternatively portrait
]{article}

\input{structure.tex} % Input the file specifying the document layout and structure
\usepackage{hyperref}


%----------------------------------------------------------------------------------------
%	ARTICLE INFORMATION
%----------------------------------------------------------------------------------------

\articletitle{Verbal Disputes} % The title of the article
\articlecitation{\cite{Chalmers2011}} % The BibTeX citation key from your bibliography

\datenotesstarted{December 19, 2018} % The date when these notes were first made
\docdate{\datenotesstarted; rev. \today} % The date when the notes were lasted updated (automatically the current date)
\docauthor{Summarized and Commnted by Shimpei Endo} % Your name

%----------------------------------------------------------------------------------------

\begin{document}

\pagestyle{myheadings} % Use custom headers
\markright{\doctitle} % Place the article information into the header

%----------------------------------------------------------------------------------------
%	PRINT ARTICLE INFORMATION
%----------------------------------------------------------------------------------------

\thispagestyle{plain} % Plain formatting on the first page

\printtitle % Print the title

%----------------------------------------------------------------------------------------
%	ARTICLE NOTES
%----------------------------------------------------------------------------------------

\section*{In a nutshell... }
In this long paper,
Chalmers discusses two dimensions of verbal disputes: methodological level and first-order level. Primally Chalmers outlines methodological perspertive to verbal disputes: criteria for verbal disputes and their consequences (section 2-6).
Chalmers then derives several impact to first-order arguments (section 7-9).
By doing so, Chalmers proposes a method to utilize verbal disputes for philosophical progress, which share the same spirit as Carnapian logical positivists (section 10).

\noindent \textbf{Keywords:} verbal disputes, disagreement, language, methodology, metaphilosophy, Carnap

\section*{Comments by me}
This paper should be one of the sources of
\cite{Cappelen2018FixingLanguage}'s standpoint.

\section{Introduction}
\paragraph{What are verbal disputes?}
``Intuitively, a dispute between two parties is verbal when the two parties agree on the relevant facts about a domain of concern and just disagree about the language used to describe that domain.''
[p. 515]

\paragraph{The oldest example: William James}

  \begin{quote}
    Which party is right depends on what you practically mean by `going round' the squirrel. If you mean passing from the north of him to the east, then to the south, then to the west, and then to the north of him again, obviously the man does go round him, for he occupies these successive positions. But if on the contrary you mean being first in front of him, then on the right of him then behind him, then on his left, and finally in front again, it is quite as obvious that the man fails to go round him ... . \emph{Make the distinction, and there is no occasion for any farther dispute.} ( James 1907, 25, Endo's emphasis)
  \end{quote}

Setting aside James' own analysis on the expression `going round', we learn a merit: philosophical disputes can evaporate into verbal disputes.

\paragraph{When words matter.}
Words matter when ``something important rests on matters of linguistic usage'' [p. 516] such as philosophy of language/linguistic, literary criticism/history (how words are used by \emph{them}?), psychology/philosophy of mind (as an evidence of mental activities).

Or, words matter when they involves practical impacts. For instance, consider law (on `mariage' or `murder'), contract, promise.
Words matter especially when they have something to do with \emph{normative} judgement (e.g. tortue, terrorism).

\paragraph{When words do not matter.}
However, if our concern aims at a first-order domain, ``a verbal dispute is \emph{mere} verbal dispute''.
``In effect, they [verbal disputes] are obstacles that we do better to move beyond, in order that we can focus on the substantive issues regarding a domain.'' [p. 517, his emphasis]

\paragraph{Two motivations for verbal disputes.}
\begin{enumerate}
  \item As a philosophical method (second-order). Philosophical discussions more or less include phases of verbal disputes. A research on verbal disputes also help philosophical advance as a \emph{tool}.
  \begin{quote}
      My own view is that the diagnosis of verbal disputes has the potential to serve as a sort of \emph{universal acid} in philosophical discussion, either dissolving disagreements or boiling them down to the fundamental disagreements on which they turn. [p.517, emphasis mine]
  \end{quote}

  \item For a first-order. To study the nature of verbal disputes reveals the nature of several important notions (such as concept, language and meaning).
\end{enumerate}

\paragraph{Construction.}
This paper is divided into two parts.
The earlier part (section 2 - 6) discusses the meta-level argument. Chalmers characterizes verbal disputes, offer a method of distinguishing them, and discusses consequences to philosophical methodology.
In the latter part (secion 7 -9), based on the framework suggest in the earlier part, Chalmers extracts ``interesting and surprising'' results in first-order philosophy.

\section{What Is a Verbal Dispute?}
[p.518]
\section{Resolving Verbal Disputes}
[p.526]
\section{The Method of Elimination within Philosophy}
[p. 530]
\section{A Possible Reaction}
[p. 535]
\section{Conceptual Analysis and Ordinary Language Philosophy}
[p. 538]
\section{Bedrock Disputes}
[p. 543]
\section{Bedrock Concepts}
[p. 549]
\section{Analyticity and Translucency}
[p. 557]
\section*{10  Carnapian Conclusion}
[p. 563]
Chalmers believes that the discussions given in this paper would revive Carnapian project. This closing section titled so points out four similarities between Chalmers and Carnap.

\paragraph{Conceptual analysis.}
First, both Carnap and Chalmers devote themselves to conceptual analysis, which specifies ``a distincitive class of primitive concepts, as well as something that can play the role of an analytic / synthetic distinction.'' (p. 563) Chalmers' counterpart of primitive truths in Carnap's Aufbau is bedrock truths (section 8).

\paragraph{Logical empiricalism.}
Second, Chalmer's framework features logical empiricists' veritificationism (cf. Carnap 1936) as its special variant (p. 565).

\paragraph{Pragmatism.}
Thirs, Carnapian pragmatism reappears in Chalmers' conceptual framework. We should look at the roles we need instead of worlds and their expressions.

\paragraph{Philosophical progress.}
Finally, both aim at ``philosophical progress''. Carnap and his logical buddies tried to resolve philosophical disputes ``once and for all'' (p. 564). Chalmers evaluates his own method more modestly but still claims that it has the potential to make clear most and solve some philosophical philosophical disputes.

%----------------------------------------------------------------------------------------
%	BIBLIOGRAPHY
%----------------------------------------------------------------------------------------

%\renewcommand{\refname}{Reference} % Change the default bibliography title

\bibliography{Mendeley} % Input your bibliography file

%----------------------------------------------------------------------------------------

\end{document}
