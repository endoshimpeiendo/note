%%%%%%%%%%%%%%%%%%%%%%%%%%%%%%%%%%%%%%%%%
% Article Notes
% LaTeX Template
% Version 1.0 (1/10/15)
%
% This template has been downloaded from:
% http://www.LaTeXTemplates.com
%
% Authors:
% Vel (vel@latextemplates.com)
% Christopher Eliot (christopher.eliot@hofstra.edu)
% Anthony Dardis (anthony.dardis@hofstra.edu)
%
% License:
% CC BY-NC-SA 3.0 (http://creativecommons.org/licenses/by-nc-sa/3.0/)
%
%%%%%%%%%%%%%%%%%%%%%%%%%%%%%%%%%%%%%%%%%

%----------------------------------------------------------------------------------------
%	PACKAGES AND OTHER DOCUMENT CONFIGURATIONS
%----------------------------------------------------------------------------------------

\documentclass[
10pt, % Default font size is 10pt, can alternatively be 11pt or 12pt
a4paper, % Alternatively letterpaper for US letter
twocolumn, % Alternatively onecolumn
landscape % Alternatively portrait
]{article}

\input{structure.tex} % Input the file specifying the document layout and structure
\usepackage{hyperref}


%----------------------------------------------------------------------------------------
%	ARTICLE INFORMATION
%----------------------------------------------------------------------------------------

\articletitle{Summary of ch.7 Corollaries of Externalism: Inscrutability, Lack of Control, and Anti-Luminosity} % The title of the article
\articlecitation{\cite{Cappelen2018FixingLanguage}} % The BibTeX citation key from your bibliography

\datenotesstarted{December 6, 2018} % The date when these notes were first made
\docdate{\datenotesstarted; rev. \today} % The date when the notes were lasted updated (automatically the current date)
\docauthor{Text by Shimpei Endo, Talk by Shunsuke Kishi} % Your name

%----------------------------------------------------------------------------------------

\begin{document}

\pagestyle{myheadings} % Use custom headers
\markright{\doctitle} % Place the article information into the header

%----------------------------------------------------------------------------------------
%	PRINT ARTICLE INFORMATION
%----------------------------------------------------------------------------------------

\thispagestyle{plain} % Plain formatting on the first page

\printtitle % Print the title

%----------------------------------------------------------------------------------------
%	ARTICLE NOTES
%----------------------------------------------------------------------------------------

\section*{In a nutshell... }
The previous chapter (chapter 6) defends the externalistic conceptual engineering.
This chapter (chapter 7) considers its consequences.

This chapter contains three parts.
The first (and main) part (7.1-7.5) elaborates the externalist's conceptual engineering suggested in the previous chapter.
An important principle of \emph{Inscrutable-Lack of Control-Will Keep Trying} is introduced.
The middle part (7.6) contrast Cappelen from Haslanger's project for the same goal. Haslanger is more descriptive and less revisionary than Cappelen.
The final part (7.7) counterargues three possible objections.

\noindent \textbf{Keywords:} externalism, inscrutability, lack of control, anti-luminosity, Sally Haslanger

\section*{7.1 Elaborating the Austerity Framework}
The previous chapter 6 has argued that there is no cookbook for successful revision.
This echoes his favorite principle in this book:
\begin{quote}
\textbf{Inscrutable--Lack of Control--Will Keep Trying}: The processes involved in conceptual engineering are for the most part \emph{inscrutable}, and \emph{we lack control of them}, but nonetheless we \emph{will and should keep trying}.
(p. 72, his emphasis)
\end{quote}

Slightly unpacked, observe the following five slogans.

\begin{enumerate}
  \item An epistemic point: Conceptual engineering often goes so messy that we cannot track.
  \item A metaphysical point: Conceptual engineering is out of our control.
  \item A psychological point: Conceptual engineers (will/must) never give up.
  \item A theoretical point: As a collorary of the three above, nobody can hide from conceptual change.
  \item An epistemic point (again!): Further, doing conceptual engineering is not lumious: you may do without knowing that you do.
\end{enumerate}

\section*{7.2 Elaborations 1 and 2: `Inscrutable' and `Lack of Control'}

Two factors -- several aspects of the past and other agents-- are working behind conceptual engineering being `inscrutable' (a epistemic point) and `lack of control' (a metaphysical point).

The past contains several aspects such as introdutory expressions (cf. stipulations), communicative chain, and sources of information in the past.
Other people, especially experts, play a crucial task for conceptual engineering (cf. Burge and Putnam).

These lead to the epistemic point: We cannot know (i) the information on past required nor (ii) the mechanism of reference-change.
The metaphysical point is its direct consequence. Nobody has control over reference-fixing process.



\section*{7.3 Elaboration 3: `Will Keep Trying'}
The previous two points would sound quite ``bleak and pessimistic''. But Cappelen is optimistic with a two-fold arguments.
Roughly, Cappelen insists that no manual for conceptual engineering does not concludes the end of conceptual engineering.

\section*{7.4 Elaboration 4: There Are No Safe Spaces Where We're in Control}
Message from Cappelen: do not overestimate our illusion: you have control over conceptual engineering as your life  (e.g. raising children).
For Capplen, a definition of a word (e.g. `intuition') in this book is even out of his control.
``What you say when you use that word is governed by what the word means, not by what you want it to mean. The rest of your paper is in English, not in some new language you created on page 2.'' (p. 76)



\section*{7.5 Elaboration 5: Conceptual Engineering Not Luminous}
``Engaging in conceptual engineering is not what  \cite{Williamson2002_knowledge} calls `a \emph{luminous} condition': We can do it without being aware of doing it and we can think we’re doing it when we're not doing it.''
(p. 78, citation fixed by me. )



\section*{7.6 A Sharp Contrast: Haslanger on Externalism and Amelioration}
Cappelen and \cite{Haslanger2012} share the same goal to make possible conceptual engineering (ameliorative projects) which are compatible with externalism. However, these two are very different.

\paragraph{Some points ofagreement between Haslanger and the Austerity Framework}
They agree on two aspects: revision (i.e. amelioration) as revelation and social scientists as reference-fixing experts.

\paragraph{7.6.2 A big point ofdisagreement between Haslanger and the Austerity Framework}
Haslanger's project is ultimately \emph{purely descriptive}. Her aim is to specify and reveal the extensions in reality. Externalism just assists this project for imposing normative considerations.

Cappelen is more \emph{revisionary} so more radical than Haslanger.

For example, consider the concept `marriage'.

\section*{7.7 Objections to the Austerity Framework}
 Objections include: (7.7.1) why not internalism then? (7.7.2) why does semantic value matter? (7.7.3) why not giving up conceptual engineering?

\paragraph{7.7.1 Objection 1: Isn't this just an argument for internalism?}
Did not Cappelen argue to abandon externalism and adopt internalism?
\noindent \textbf{Reply.}
Internalism does not always rescue.

\paragraph{7.7.2 Objection 2: Why care about semantic values if they are inscrutable?}
??

\noindent \textbf{Reply.}??

\paragraph{7.7.3 Objection 3: Isn't this just a debunking project?}
Some reasonably intereprets this book as actually claiming that conceptual engineering is impossible.

\noindent \textbf{Reply.} Notice that it is common for normative domains to be inscrutable and out of control. For instance, theorists, say, of theory of justice, often renew the theory without knowing how to implement.
\section*{Comments by Kishi.}
%----------------------------------------------------------------------------------------
%	BIBLIOGRAPHY
%----------------------------------------------------------------------------------------

%\renewcommand{\refname}{Reference} % Change the default bibliography title

\bibliography{Mendeley} % Input your bibliography file

%----------------------------------------------------------------------------------------

\end{document}
