%%%%%%%%%%%%%%%%%%%%%%%%%%%%%%%%%%%%%%%%%
% Article Notes
% LaTeX Template
% Version 1.0 (1/10/15)
%
% This template has been downloaded from:
% http://www.LaTeXTemplates.com
%
% Authors:
% Vel (vel@latextemplates.com)
% Christopher Eliot (christopher.eliot@hofstra.edu)
% Anthony Dardis (anthony.dardis@hofstra.edu)
%
% License:
% CC BY-NC-SA 3.0 (http://creativecommons.org/licenses/by-nc-sa/3.0/)
%
%%%%%%%%%%%%%%%%%%%%%%%%%%%%%%%%%%%%%%%%%

%----------------------------------------------------------------------------------------
%	PACKAGES AND OTHER DOCUMENT CONFIGURATIONS
%----------------------------------------------------------------------------------------

\documentclass[
10pt, % Default font size is 10pt, can alternatively be 11pt or 12pt
a4paper, % Alternatively letterpaper for US letter
twocolumn, % Alternatively onecolumn
landscape % Alternatively portrait
]{article}

%%%%%%%%%%%%%%%%%%%%%%%%%%%%%%%%%%%%%%%%%
% Article Notes
% Structure Specification File
% Version 1.0 (1/10/15)
%
% This file has been downloaded from:
% http://www.LaTeXTemplates.com
%
% Authors:
% Vel (vel@latextemplates.com)
% Christopher Eliot (christopher.eliot@hofstra.edu)
% Anthony Dardis (anthony.dardis@hofstra.edu)
%
% License:
% CC BY-NC-SA 3.0 (http://creativecommons.org/licenses/by-nc-sa/3.0/)
%
%%%%%%%%%%%%%%%%%%%%%%%%%%%%%%%%%%%%%%%%%

%----------------------------------------------------------------------------------------
%	REQUIRED PACKAGES
%----------------------------------------------------------------------------------------

\usepackage[includeheadfoot,columnsep=2cm, left=1in, right=1in, top=.5in, bottom=.5in]{geometry} % Margins

\usepackage[T1]{fontenc} % For international characters
\usepackage{XCharter} % XCharter as the main font

\usepackage{natbib} % Use natbib to manage the reference
\bibliographystyle{apalike} % Citation style

\usepackage[english]{babel} % Use english by default

%----------------------------------------------------------------------------------------
%	CUSTOM COMMANDS
%----------------------------------------------------------------------------------------

\newcommand{\articletitle}[1]{\renewcommand{\articletitle}{#1}} % Define a command for storing the article title
\newcommand{\articlecitation}[1]{\renewcommand{\articlecitation}{#1}} % Define a command for storing the article citation
\newcommand{\doctitle}{\articlecitation\ --- ``\articletitle''} % Define a command to store the article information as it will appear in the title and header

\newcommand{\datenotesstarted}[1]{\renewcommand{\datenotesstarted}{#1}} % Define a command to store the date when notes were first made
\newcommand{\docdate}[1]{\renewcommand{\docdate}{#1}} % Define a command to store the date line in the title

\newcommand{\docauthor}[1]{\renewcommand{\docauthor}{#1}} % Define a command for storing the article notes author

% Define a command for the structure of the document title
\newcommand{\printtitle}{
\begin{center}
\textbf{\Large{\doctitle}}

\docdate

\docauthor
\end{center}
}

%----------------------------------------------------------------------------------------
%	STRUCTURE MODIFICATIONS
%----------------------------------------------------------------------------------------

\setlength{\parskip}{3pt} % Slightly increase spacing between paragraphs

% Uncomment to center section titles
%\usepackage{sectsty}
%\sectionfont{\centering}

% Uncomment for Roman numerals for section numbers
%\renewcommand\thesection{\Roman{section}}
 % Input the file specifying the document layout and structure
\usepackage{hyperref}
\usepackage{enumitem}


%----------------------------------------------------------------------------------------
%	ARTICLE INFORMATION
%----------------------------------------------------------------------------------------

\articletitle{From Four- to Five-Dimensionalism} % The title of the article
\articlecitation{\cite{Graham2015}} % The BibTeX citation key from your bibliography

\datenotesstarted{December 17, 2018} % The date when these notes were first made
\docdate{\datenotesstarted; rev. \today} % The date when the notes were lasted updated (automatically the current date)
\docauthor{Summarized and Commented by Shimpei Endo} % Your name

%----------------------------------------------------------------------------------------

\begin{document}

\pagestyle{myheadings} % Use custom headers
\markright{\doctitle} % Place the article information into the header

%----------------------------------------------------------------------------------------
%	PRINT ARTICLE INFORMATION
%----------------------------------------------------------------------------------------

\thispagestyle{plain} % Plain formatting on the first page

\printtitle % Print the title

%----------------------------------------------------------------------------------------
%	ARTICLE NOTES
%----------------------------------------------------------------------------------------

\section*{In a nutshell... }
Graham defends five-dimensinoalism for modality with aid of analogy to four-dimensionalism for time.
Graham recycles the strategy which already has been established for four-dimensionalsm: we should adopt it because it is helpful to solve puzzles.
This article contains a summary of four-dimensionalism and five-dimensinoalism.
Graham focuses on the issues of \emph{coincedence} (sec. 2), which appears both in discussions of time and of modality.

\noindent \textbf{Keywords:} modal realism, dimension, four-dimensionalism, Ted Sider, five-dimensinalism

\section*{Comments by me}
Why not Yagisawa?
Cite his criticism on Lewis' counterpart theory (p.19) as a rare sample for supporting modal realism in general but going against the proclaimed founder David Lewis.

\section*{Unsectioned Introduction*}
\paragraph{Identity over time and identity across worlds are similar.}
Five-dimensinoalsm claims that ``objects are modally extended and their trans-world identity is a matter of having distinct modal parts located in different possible worlds.'' (p. 14-15)
Graham claims it is unfair that five-dimensionalism has received fewer attention compared to its successful analogue in time.

\section{Four-Dimensionalism and Five-Dimensionalism}
\paragraph{Overview Four-dimensionalism.}
Four-dimensionalism's two fundamental claims are:
(i) objects have temporal parts and (ii) objects exist at different times by having different temporal parts located at different times. (p. 15, \cite{Sider2001})

Four-dimensinoalism offers a simple and working answer to puzziling phenomena, borrowing resources of parts from space. For example, four-dimensinoalists can easily explain how both claims are compatible for the identical entity: I was a kid and I am not. A temporal part of me (say when I was six years old) is a kid while another (say, at this moment) is not.

\paragraph{Five-dimensinoalism recycles four-.}
Five-dimensinoalists apply four-dimensinalists' argument to define identity across possible worlds.
Consider: Socrates is wise in this actual world but not necessarily (i.e. he could have been stupid). According to five-dimensinoalism, Socrates has a modal part which is wise (happen to be in this world) and another modal part which is stupid.
Five-dimensionalists assign \emph{modal worms} to metaphysical status of ordinary objects.

\paragraph{Five-dimensionalism is compatible with actualism.}
Modal actualism claims that everything is actual.
Graham argues that nothing inconsistent arises between five-dimensionalism and actualism (p. 17). Notice that an actualist can affirm that Socrates exists in other worlds (but such worlds are, according to actualism, members of our actual world).
This justification goes pararell with four-dimensinoalism and presentism.

\paragraph{Five-dimensionalists' further questions:}
What conditions are needed for being a modal part of Socrates (and not of Plato)?

\paragraph{Compare five-dimensionalism and David Lewis' counterpart theory.}
(p. 18-19)
The temporal analogue of five-dimensinoalism and counterpart theory is the combination of four-dimensionalism and \emph{stage view}.
As a four-dimensinoalist and a stage-theoriest do, a five-dimensionalis and a counterpart theorist disagree on the \emph{nature of ordinary objects}.

Graham is against the pair of stage-theory (4D) and counterpart theory (5D) for two reasons.
(i) First, the stage-couterpart set is inconsistent. (ii) Second, their selecting a couterpart of a given object is arbitrary (a matter of convention).

\section{The Puzzles of Temporary Coincidence and Contingent Coincidence}
We should accept four-dimensinoalism because it solves many philosophical puzzles about coincidence with a theoretically beautiful manner. This reasoning should apply to five-dimensionalism. [p. 20]

\paragraph{Two types of coincidence: temporary and contingent.}
[p.20]

\paragraph{Why coincidence matters?}
Because it crash with our intuition for material objects such that they cannot occupy the same space.
A classical instance supporting this ordinary view is John Locke, writing:
``where we imagine any space taken up by a solid substance, we conveive it so to possess it, that it excludes all other solid substances,'' and
``This resistance, whereby it keeps other bodies out of the space which it possesses, is so great, that no force, how great soever, can surmount it''.
(John Locke, \textit{An Essay Concerning Human Understanding},  edited by Roger Woolhouse, London: Penguin Classics, 2004, p.125)

\paragraph{Overlaps do not matter.}
In contrast, anohter concept  {overlap} seems similar to coincidence but does not matter much. For instance, consider two loads intersect each other. The interaction part is, acceptably, part of both of two roads (\emph{mereological overlap}). Or, think about a city (say, Kamakura: my example) can be a part of a province (Kanagawa) and a country (Japan) at the same time with no problem at all.

\paragraph{How five/four-dimensionalism solve the coincidence issue?}
Four-dimensinoalists \emph{reduce} coincidence into temporal overlap or sharing the temporal parts. Graham appeals this is a huge advantage of four-dimensionalim for eliminating a puzzilin concept (coincidence) to a more comprehensive and acceptable one (overlap).
Five-dimensionalists provide the same virtue for the modal case: contingent coincidence.
Five-dimensinoalism explains the coincidence as sharing \emph{modal} parts.

\paragraph{Details.}
To support the general claim given above,
Graham offers several particular cases. Each case deals with both temporal and modal versions.
\begin{enumerate}[label=(\alph*)]
  \item Constitution (p.9).
  \item Undetached Parts (p.10).
  \item Fission and Fusion (p.12).
\end{enumerate}

\section{Conclusion}
Graham argues that we should believe five-dimensinoalism (for modality) following the same criterion for the defense of four-dimensionalism: as a promising direction to solve philosophical puzzles.
From Graham's perspective, modal philosophers have not paid enough attention to five-dimensinoalism. Graham has offered an example case of \emph{(contingent) coincidence}, in which five-dimensinalists can copy the argument of her metaphysical causin four-dimensinoalism.


%----------------------------------------------------------------------------------------
%	BIBLIOGRAPHY
%----------------------------------------------------------------------------------------

%\renewcommand{\refname}{Reference} % Change the default bibliography title

\bibliography{Mendeley} % Input your bibliography file

%----------------------------------------------------------------------------------------

\end{document}
