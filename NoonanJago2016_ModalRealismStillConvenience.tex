%%%%%%%%%%%%%%%%%%%%%%%%%%%%%%%%%%%%%%%%%
% Article Notes
% LaTeX Template
% Version 1.0 (1/10/15)
%
% This template has been downloaded from:
% http://www.LaTeXTemplates.com
%
% Authors:
% Vel (vel@latextemplates.com)
% Christopher Eliot (christopher.eliot@hofstra.edu)
% Anthony Dardis (anthony.dardis@hofstra.edu)
%
% License:
% CC BY-NC-SA 3.0 (http://creativecommons.org/licenses/by-nc-sa/3.0/)
%
%%%%%%%%%%%%%%%%%%%%%%%%%%%%%%%%%%%%%%%%%

%----------------------------------------------------------------------------------------
%	PACKAGES AND OTHER DOCUMENT CONFIGURATIONS
%----------------------------------------------------------------------------------------

\documentclass[
10pt, % Default font size is 10pt, can alternatively be 11pt or 12pt
a4paper, % Alternatively letterpaper for US letter
twocolumn, % Alternatively onecolumn
landscape % Alternatively portrait
]{article}

%%%%%%%%%%%%%%%%%%%%%%%%%%%%%%%%%%%%%%%%%
% Article Notes
% Structure Specification File
% Version 1.0 (1/10/15)
%
% This file has been downloaded from:
% http://www.LaTeXTemplates.com
%
% Authors:
% Vel (vel@latextemplates.com)
% Christopher Eliot (christopher.eliot@hofstra.edu)
% Anthony Dardis (anthony.dardis@hofstra.edu)
%
% License:
% CC BY-NC-SA 3.0 (http://creativecommons.org/licenses/by-nc-sa/3.0/)
%
%%%%%%%%%%%%%%%%%%%%%%%%%%%%%%%%%%%%%%%%%

%----------------------------------------------------------------------------------------
%	REQUIRED PACKAGES
%----------------------------------------------------------------------------------------

\usepackage[includeheadfoot,columnsep=2cm, left=1in, right=1in, top=.5in, bottom=.5in]{geometry} % Margins

\usepackage[T1]{fontenc} % For international characters
\usepackage{XCharter} % XCharter as the main font

\usepackage{natbib} % Use natbib to manage the reference
\bibliographystyle{apalike} % Citation style

\usepackage[english]{babel} % Use english by default

%----------------------------------------------------------------------------------------
%	CUSTOM COMMANDS
%----------------------------------------------------------------------------------------

\newcommand{\articletitle}[1]{\renewcommand{\articletitle}{#1}} % Define a command for storing the article title
\newcommand{\articlecitation}[1]{\renewcommand{\articlecitation}{#1}} % Define a command for storing the article citation
\newcommand{\doctitle}{\articlecitation\ --- ``\articletitle''} % Define a command to store the article information as it will appear in the title and header

\newcommand{\datenotesstarted}[1]{\renewcommand{\datenotesstarted}{#1}} % Define a command to store the date when notes were first made
\newcommand{\docdate}[1]{\renewcommand{\docdate}{#1}} % Define a command to store the date line in the title

\newcommand{\docauthor}[1]{\renewcommand{\docauthor}{#1}} % Define a command for storing the article notes author

% Define a command for the structure of the document title
\newcommand{\printtitle}{
\begin{center}
\textbf{\Large{\doctitle}}

\docdate

\docauthor
\end{center}
}

%----------------------------------------------------------------------------------------
%	STRUCTURE MODIFICATIONS
%----------------------------------------------------------------------------------------

\setlength{\parskip}{3pt} % Slightly increase spacing between paragraphs

% Uncomment to center section titles
%\usepackage{sectsty}
%\sectionfont{\centering}

% Uncomment for Roman numerals for section numbers
%\renewcommand\thesection{\Roman{section}}
 % Input the file specifying the document layout and structure
\usepackage{hyperref}


%----------------------------------------------------------------------------------------
%	ARTICLE INFORMATION
%----------------------------------------------------------------------------------------

\articletitle{Modal Realism, Still at Your Convenience} % The title of the article
\articlecitation{\cite{NoonanJago2016_ModalRealismStillConvenience}} % The BibTeX citation key from your bibliography

\datenotesstarted{December 26, 2018} % The date when these notes were first made
\docdate{\datenotesstarted; rev. \today} % The date when the notes were lasted updated (automatically the current date)
\docauthor{Summarized and Commnted by Shimpei Endo} % Your name

%----------------------------------------------------------------------------------------

\begin{document}

\pagestyle{myheadings} % Use custom headers
\markright{\doctitle} % Place the article information into the header

%----------------------------------------------------------------------------------------
%	PRINT ARTICLE INFORMATION
%----------------------------------------------------------------------------------------

\thispagestyle{plain} % Plain formatting on the first page

\printtitle % Print the title

%----------------------------------------------------------------------------------------
%	ARTICLE NOTES
%----------------------------------------------------------------------------------------

\section*{In a nutshell... }
Noonan and Jago defends Lewis's modal realism from Divers (2014), which claims that some (say, spatiotemporal) \textit{de re} modal truths are inconvenient for Lewisian theory.


\noindent \textbf{Keywords:} modal realism, pro-Lewis

\section*{Comments by me}

\section{Introduction}
\paragraph{Construction.}
[p. 299]
First Noonan and Jago summarize the objection of Divers (2014) (section 2). Then they argue that Lewis followers do not have to read de re modality as Divers requests, citing Lewis (1986)'s own words (section 3). Lastly, the authors respond to possible objections from Divers.

\section{Divers's Argument}
\paragraph{Lewisian worlds: spatially connected for reduction. }
The ``tenet'' of Lewis's modal realism is the metaphysical nature of possible worlds: they are maximally spatiotemporally connected entities. Thanks to this metaphysical characterization, Lewis can reduce modal notions into non-modal (i.e. spatiotemporally connected) terms while his rivals cannot.

\paragraph{Divers's horn: Lewis cannot reduce neither!}
Divers argues that Lewis is also unsuccessful in reduction.  According to Divers, spatiotemporal relations hold across different worlds.
Diver's example is:

\noindent (1) Usain might have been taller than he actually is.

Lewis interprets this modal notion in terms of counterparts of Usain in worlds different from ours. Divers here accuses that Lewis spatially compares two things in different worlds. This contradicts himself, for a world should be maximal with respect to spatial relation.

Divers offers further examples for sophistication. One of them is:

\noindent
(2) It is true of the tallest actual thing that it might have been taller (than it actually is).

Upon these sophiscated examples of de re modality which spatially compare things in distinct worlds, Divers concludes that ``these modal truths are \emph{inconvenient} to the Lewisian'' . [p. 301, my emphasis]


\section{No Inconvenience}

\paragraph{Ohter interpretations available. }
Noonan and Jago reject such inconvenience by specifying that Diver's argument an unsaid assumption: there is no other plausible interpretations available.
Noonan and Jago offer particular interpretations which disagree with Divers' assumption:

($2_{\text{dup}}$) The tallest actual thing t has a counterpart, c, in world w, which is taller than a duplicate of t in w.
The tallest actual thing t has a counterpart, c, in world w, which is taller than a duplicate of t in w.

($2_{\text{cp}}$) The tallest actual thing t has counterparts, c1 and c2, in world w, such that c1 is taller than c2.
The tallest actual thing t has counterparts, c1 and c2, in world w, such that c1 is taller than c2.

\paragraph{Lewis backs up. }
Guranteed by Lewis's own words:

\begin{quotation}
  \noindent Things that are parts of two worlds may be simultaneous or not, they may be in the same or different towns, they may be near or far from one another, in very natural counterpart-theoretic senses. But these are not genuine spatio-temporal relations across worlds. The only trans-world relations involved are internal relations of similarity; not indeed be- tween the very individuals that are quasi-simultaneous (or whatever), but between larger duplicate parts of the two worlds wherein those individuals are situated. (Lewis 1986: 71)
\end{quotation}

This passage seems to suggest to regard a `counterpart' and a `duplicate' as `quasi-spatiotemporal' relations.

\paragraph{Noonan-Jago's interpretations are plausible too. }
Noonan and Jago further supports this interpreatation is not only available but also plausible. It is charity which forces us to read in Noonan-Jago-Lewis's way instead of Divers's.

\section{Actually?}
This closing section responses to a possible objection from Divers and his followers [p.303].

\begin{quotation}
A Lewisian counterpart theory is \emph{not a translation manual}, from English
(or modal logic) to the extensional language of counterparts. (Perhaps Lewis (1968) can be read that way, in making the point that counterpart theory is at least as expressive as QML. But Lewis is clear that this is not how to read Lewis (1986).) As Lewis says, we must always \emph{‘interpret the message to make it make sense’}. And to do so, in our examples, one must interpret via a duplicate (or otherwise a good counterpart) at the world in question.
[p. 303, my emphasis]
\end{quotation}

%----------------------------------------------------------------------------------------
%	BIBLIOGRAPHY
%----------------------------------------------------------------------------------------

%\renewcommand{\refname}{Reference} % Change the default bibliography title

\bibliography{Mendeley} % Input your bibliography file

%----------------------------------------------------------------------------------------

\end{document}
