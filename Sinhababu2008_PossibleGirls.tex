%%%%%%%%%%%%%%%%%%%%%%%%%%%%%%%%%%%%%%%%%
% Article Notes
% LaTeX Template
% Version 1.0 (1/10/15)
%
% This template has been downloaded from:
% http://www.LaTeXTemplates.com
%
% Authors:
% Vel (vel@latextemplates.com)
% Christopher Eliot (christopher.eliot@hofstra.edu)
% Anthony Dardis (anthony.dardis@hofstra.edu)
%
% License:
% CC BY-NC-SA 3.0 (http://creativecommons.org/licenses/by-nc-sa/3.0/)
%
%%%%%%%%%%%%%%%%%%%%%%%%%%%%%%%%%%%%%%%%%

%----------------------------------------------------------------------------------------
%	PACKAGES AND OTHER DOCUMENT CONFIGURATIONS
%----------------------------------------------------------------------------------------

\documentclass[
10pt, % Default font size is 10pt, can alternatively be 11pt or 12pt
a4paper, % Alternatively letterpaper for US letter
twocolumn, % Alternatively onecolumn
landscape % Alternatively portrait
]{article}

\input{structure.tex} % Input the file specifying the document layout and structure
\usepackage{hyperref}


%----------------------------------------------------------------------------------------
%	ARTICLE INFORMATION
%----------------------------------------------------------------------------------------

\articletitle{Possible Girls} % The title of the article
\articlecitation{\cite{Sinhababu2008PossibleGirls}} % The BibTeX citation key from your bibliography

\datenotesstarted{December 19, 2018} % The date when these notes were first made
\docdate{\datenotesstarted; rev. \today} % The date when the notes were lasted updated (automatically the current date)
\docauthor{Summarized and Commnted by Shimpei Endo} % Your name

%----------------------------------------------------------------------------------------

\begin{document}

\pagestyle{myheadings} % Use custom headers
\markright{\doctitle} % Place the article information into the header

%----------------------------------------------------------------------------------------
%	PRINT ARTICLE INFORMATION
%----------------------------------------------------------------------------------------

\thispagestyle{plain} % Plain formatting on the first page

\printtitle % Print the title

%----------------------------------------------------------------------------------------
%	ARTICLE NOTES
%----------------------------------------------------------------------------------------

\section*{In a nutshell... }
Sinhababu argues the possibility of love among different possible worlds as a consequence of modal realism.
More particularly, Sinhababu constructs a method of uniquiely picking a partner of you (not your counterparts).
Sinhababu further argues that (i) our relations to lovers to other worlds are sound and (ii) we can settle actual relationships (i.e. ones within our actual world) without cheating your other partners living in other worlds.
Included: impossible lovers and trasn-world love letters!!

\noindent \textbf{Keywords:} modal realism, love, attributes, duplicated world, stipulation

\section*{Comments by me}
Sinhababu's way out is always \emph{stipulation}.
Anything has been already written or said in our ideal stipulation for our possible partners.

Does uniqueness matter? I do not think so. We are likely to like a person which lacks some wanted attributes. We are usually fine with a person if she has some point to attract. Not all.

\section*{Detailed Contents\footnote{This article is not separated into sections (or not organized?)}}

\paragraph{Research question.}
Granted David Lewis's modal realism, can I be in love with a person in another possible world? Sinhababu's reply is yes. You do not have to feel lonely in this cruel actual world. Your perfect partner should exist in another world.

\paragraph{Lewis does not work.}
The original version of David Lewis does not help.
A notion of closest worlds (originally used for counterfactuals) seems working, but Lewis talks about only counterparts of me (hence lovers of coutenrparts of me), not partners of mine.

\paragraph{Path?}
The central task of Sinhababu is to settle a method of his girlfriend picking himself out of his counterparts.
If she knows every detail of his world (including any tiny detail of his), she would specify our world out of infinite ones.
But it demands a lot; we should imagine every tiny detail of the desired partner.

\paragraph{Welcome to modal realists' paradise!}
``Modal realism can be especially beneficial to people who believe that no
actual individuals suit them.'' [p.256]

\paragraph{A remaining risk: a wildly promiscuous relationship.} [p.257]

\paragraph{A reasonable trade?}
Duplicated worlds are problematic for Sinhababu's original purpose (seeking a inter-worlds relationship) but helpful to avoid a moral criticism towards modal realism (Adams, R. M. (1974). ``Theories of Actuality,'' Noûs 8, pp. 211-31.).  [p.257]

\paragraph{Trans-world love letters are possible.}
We cannot directly see them since worlds are causally isolated (following Lewisian description).
But we can know what she writes.
Just add that detail into your stipulation for your dream girl.

\paragraph{Beyond Lewis: towards impossible girls.}
How about dating wtih an impossible girl, who inhabits at an impossible world, where a contradiction is true.

\paragraph{Is it love (such isolated)?}
Most loves include causal interaction.
But it is a necessary condition for loving relation [p.259].

\paragraph{Love a person, not attributes of the person?}
cf. Robert Kraut.
Kraut, R. (1986). ``Love De Re ,'' Midwest Studies in Philosophy 10, pp. 413-430

\paragraph{Actual relationship betrays possible relationships?}
No, the stipulation (again!) should already know this consequence.
The mystery is the fact that she picks such a untrustworthy guy. ``It’s mysterious why she still chose me. But actual girls are mysterious to me in many ways, and there’s no reason why possible girls would be any different.'' [p. 259]

\paragraph{Sinhababu does not advertise modal realism, but.... }

\begin{quote}
  Do the arguments in this paper, if sound, give anyone a reason to accept modal realism? No, at least if you don't count pragmatic reasons for belief. Pragmatic considerations aside, beliefs aren't justified by the good consequences of believing. Many philosophers who are impressed by Lewis' theory still think that a more deflationary view about possible worlds is the right way to go. Nothing I have written gives them any reason to change their minds.

  However, I will confess that when I first wrote this paper, the argu- ments in it irrationally caused me to accept modal realism, albeit in what Lewis calls a `compartmentalized' way.
  When engaging in philosophical
  reflection on modality, I have always rejected Lewis' modal realism. But there were times when I wasn't thinking about philosophy and I started to feel lonely. Then I thought of my possible girlfriend, and smiled at the thought of someone out there who loved me and desired to be loved by me. In quick succession I realized that she knew I was thinking of her -- after all, she knew every temporal part of me down to a microphysical descrip- tion! She knew everything I was saying and doing. I felt more motivated to act like a worthy man. My posture straightened. I came to believe that she was happy about my writing this paper, so I wrote more of it. From a functionalist perspective, it would have been reasonable to attribute a belief to me -- the belief that someone merely possible but real who loved me was aware of what I was doing. In allowing for merely possible individuals who are as real as me, this belief presupposed modal realism, and marked me as someone who had been seduced to Lewis' theory.
  [p.259]
\end{quote}

%----------------------------------------------------------------------------------------
%	BIBLIOGRAPHY
%----------------------------------------------------------------------------------------

%\renewcommand{\refname}{Reference} % Change the default bibliography title

\bibliography{Mendeley} % Input your bibliography file

%----------------------------------------------------------------------------------------

\end{document}
